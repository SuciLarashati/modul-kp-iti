\documentclass[../main.tex]{subfiles}

\begin{document}
\chapter{Mengenal Bahasa Pemrograman C++}
\section{Fakta Tentang Bahasa C++}
C++ adalah sebuah bahasa pemrograman tingkat menengah yang diciptakan pada tahun
1979 oleh Bjarne Stroustrup, yang awalnya diciptakan sebagai ekstensi dari
bahasa pemrograman C.

Segala hal dalam C++ itu \eng{case sensitive}: \code{username} tidak sama
dengan \code{UserName}.

\section{Program C++ sederhana}
Program menggunakan bahasa C++ biasanya diakhiri dengan ekstensi \enquote{.cpp};

\begin{figure}
  \centering
  \begin{tikzpicture}
    \node [rectangle, draw] (code)                           {Sumber Kode};
    \node [circle, draw]    (compiler) [right = of code]     {\emph{Compiler}};
    \node [rectangle, draw] (program)  [right = of compiler] {File Binari Program};
    \draw [->, >=stealth, thick] (code) -- (compiler);
    \draw [->, >=stealth, thick] (compiler) -- (program);
  \end{tikzpicture}
  \caption{Gambar proses kompilasi}
  \label{comp-proc}
\end{figure}

Contoh program sederhana:
\cppfile{code/hello.cpp}

\begin{table}
\centering
\begin{tabular}{@{} l l @{}}
  \toprule
  \emph{Escape Sequence}  & Karakter yang direpresentasikan  \\
  \midrule
  \textbackslash{}a    & Bell sistem (suara \emph{beep})\\
  \textbackslash{}b    & \emph{Backspace} (hapus)\\
  \textbackslash{}f    & \emph{Page Break}\\
  \textbackslash{}n    & Baris baru\\
  \textbackslash{}r    & Mengembalikan kursor ke awal baris\\
  \textbackslash{}t    & Tab\\
  \textbackslash{}\textbackslash{}    & \emph{Backslash}\\
  \textbackslash{}'    & Tanda kutip satu\\
  \textbackslash{}"    & Tanda kutip dua\\
  \bottomrule
\end{tabular}
\caption{\eng{Escape Sequence} yang dapat digunakan}
\label{tipe-escape}
\end{table}

Keterangan perbaris:
\begin{enumerate}
  \item \code{/*\ldots*/} menandakan bahwa apapun setelah \code{/*} sampai
  \code{*/} adalah sebuah komentar. Selain itu, dapat juga menulis komentar
  dengan mengawali baris dengan \code{\slash{}\slash{}}.
  Komentar tidak akan diproses.
  \item Baris yang diawali \code{\#} biasa disebut perintah
  \eng{preprocessor}, yang biasanya dapat mengganti kode yang akan
  di kompilasi. \code{\#include} menandakan bahwa sebelum melakukan kompilasi
  harus memasukkan kode dalam pustaka \emph{iostream}.
  \item \code{int main() \{\ldots\}} fungsi utama sebuah program berjalan.
  Kurung kurawal menandakan pengelompokan beberapa perintah dalam satu blok.
  \item
	\begin{itemize}
	\item \code{cout <<} adalah syntax yang digunakan untuk menampilkan teks
	  ke layar.
	\item Kata yang diapit oleh tanda kutip dua, seperti
	  \code{"Hello World!\textbackslash{}n"}, itu biasa disebut dengan
	  \eng{string} dalam bahasa pemrograman.
	\item \code{\textbackslash{}n} biasa disebut sebagai
	  \eng{escape sequence}. \eng{Escape sequence}
	  adalah simbol yang memiliki arti khusus dalam \eng{string}.
	  Lihat tabel \ref{tipe-escape}.
	\end{itemize}
  \item \code{return 0} memberitahukan ke sistem operasi program berjalan
  dengan lancar.
\end{enumerate}

Setiap baris dalam bahasa pemrograman C++ perlu diakhiri dengan tanda titik koma
\enquote{\code{;}}. Kecuali, untuk \eng{preprocessor} dan
tanda akhiran blok \{\}.

\section{Fitur Dasar Program}
\subsection{Nilai dan Pernyataan}
Sebelum praktikum ini berlanjut, ada beberapa definisi yang harus praktikan
kuasai;

\begin{description}
\item[Pernyataan] adalah sebuah unit dalam kode yang melakukan sesuatu -- sebuah
  dasar dari pembuatan program.
\item[Ekspresi] adalah pernyataan yang memiliki \emph{nilai} -- misalnya angka,
  karakter, hasil penjumlahan angka, atau kalimat. \(5 + 6 / 3\) dan
  \code{"Hello World!\textbackslash{}n"} adalah ekspresi.
\end{description}

Perlu diingat sekali lagi, bahwa tidak semua \emph{pernyataan} adalah
\emph{ekspresi}.

\subsection{Operasi Aritmatika}
Kita dapat melakukan operasi matematika dalam program kita. Sebagai contoh,
ganti \code{"Hello World!\textbackslash{}n"} dengan \(5 + 6 / 3\). Sehingga yang
muncul di layar adalah angka \(7\).

\begin{minted}{c++}
cout << 5 + 6 / 3;
\end{minted}

Operator yang dapat digunakan sama seperti pada matematika biasa. \(+\), \(-\),
\(*\), \(/\), dan tanda kurung \(()\) memiliki arti yang sama seperti operator matematika
pada umumnya. Hanya saja, ada tambahan operator \(\%\) sebagai operator operasi
modulo atau sisa hasil bagi.

\subsection{Tipe Data}
Setiap ekspresi dalam bahasa pemrograman C++ memiliki sebuah tipe data. Tipe
data bisa diartikan sebagai jenis dari data tersebut. Contoh, \(42\) adalah
\emph{integer}, \(3.14\) adalah angka \eng{floating-point} (desimal), dan
\code{'A'} adalah karakter. Setiap tipe data memiliki perbedaan memori yang
digunakan untuk menyimpannya.

Sebuah operasi pada tipe data hanya dapat dilakukan dengan tipe data yang sesuai.
Sebuah operator juga akan menghasilkan nilai yang sesuai dengan tipe dari operan
yang digunakan dalam operasi. Misalnya, \(1/4\) hasilnya akan dibulatkan menjadi
integer \(0\). Untuk mendapatkan hasil \(0.25\), kita harus melakukan operasi
menggunakan angka desimal, sehingga operasi \(1/4\) diubah menjadi \(1.0/4.0\).

\begin{table}
\centering
\begin{tabular}{@{} l l @{}}
  \toprule
  Jenis  & Deskripsi  \\
  \midrule
  int    & Bilangan Bulat\\
  float  & Bilangan Desimal\\
  char   & Sebuah Karakter, ditandakan dengan tanda kutip satu ('a', '3')\\
  string & Kalimat/Kumpulan Karakter ("ITI")\\
  bool   & Nilai Boolean, yaitu \emph{True} (\(1\)) atau \emph{False} (\(0\))\\
  \bottomrule
\end{tabular}
\caption{Tipe data standar yang sering digunakan}
\label{tipe-var}
\end{table}

\section{Variabel}
Variabel adalah sebuah tempat yang memiliki nama yang berguna untuk menyimpan
sebuah nilai. Sebuah variabel biasanya memiliki jenis\slash{}tipe tersendiri
(lihat tabel \ref{tipe-var}). Fungsi variabel adalah untuk menyimpan nilai
sehingga kita bisa memanggil nilai itu lagi.

Contoh, kita akan menggunakan hasil dari \(3+3\) berulang kali. Kita bisa
menyimpan hasil operasi tersebut dalam sebuah variabel yang akan kita sebut
variabel \code{x}.

\cppfile{code/variabel.cpp}

\paragraph{Catatan:}
Perhatikan kita bisa menyambungkan \code{<<} dalam \code{cout} untuk menampilkan
lebih dari satu ekspresi atau variabel.

Baris ke-6 biasa disebut sebagai deklarasi variabel \code{x}. Kita harus
memberitahu kepada \eng{compiler} jenis dari variabel \code{x},
sehingga komputer tahu berapa memori yang harus dialokasikan dan juga operasi
apa saja yang bisa dilakukan kepada variabel tersebut.

Baris ke-7 biasa disebut sebagai inisialisasi variabel \code{x}, dimana kita
memberikan nilai awal untuk \code{x}. Di baris ini, kita juga berjumpa dengan
operator \(=\), yang digunakan untuk memberi nilai pada variabel. Kita juga bisa
mengubah nilai \code{x} dengan menggunakan operator \(=\) lagi.

Kita dapat menyederhanakan program di atas, dengan menggabungkan baris ke-6 dan
ke-7:
\begin{minted}{cpp}
int x = 3 + 3;
\end{minted}

Bentuk deklarasi\slash{}inisialisasi seperti ini terlihat lebih rapih, sehingga
jika bisa bentuk ini selalu praktikan gunakan.

\begin{table}[htb]
\centering
\begin{tabular}{@{} l l l l @{}}
  \toprule
  auto & else & namespace & switch\\
  bool & enum & new & this\\
  break & explicit & operator & throw\\
  case & extern & private & true\\
  catch & false & public & typedef\\
  char & float & return & union\\
  class & for & short & unsigned\\
  const & goto & signed & using\\
  continue & if & sizeof & virtual\\
  default & int & static & void\\
  do & long & struct & while\\
  \bottomrule
\end{tabular}
\caption{Contoh \emph{Keyword} dalam bahasa pemrograman C++}
\label{ex-keyword}
\end{table}

Variabel memiliki beberapa ketentuan untuk penentuan namanya;
\begin{enumerate}
  \item Variabel boleh terdiri dari karakter A--Z, a--z, 0--9, dan garis bawah.
  \item Variabel tidak boleh dimulai dengan angka, contoh; \code{3\_ayam}.
  \item Variabel tidak diperkenankan mengikuti \eng{keywords} (tabel
  \ref{ex-keyword}) dalam bahasa pemrograman C++.
\end{enumerate}

\section{Masukan (Input)}
Kita sudah dapat memanipulasi nilai. Sekarang kita bisa juga meminta pengguna
program untuk memasukkan nilai yang diinginkan ke variabel.

\cppfile{code/input.cpp}

Seperti \code{cout <<} yang digunakan untuk menampilkan nilai, \code{cin >>}
(baris ke-7) digunakan untuk memasukkan nilai. Sehingga saat pengguna mengetik
nilai integer, lalu menekan tombol \code{Enter}, nilai yang diketik akan masuk
ke dalam variabel \code{x}.

\paragraph{Latihan}
\begin{enumerate}
  \item Buat sebuah program yang dapat digunakan untuk meminta nama dan nomor
  mahasiswa, lalu tampilkan nama dan nomor mahasiswa tersebut! (gunakan variabel
  tipe \eng{string} untuk nama)
  \item Buat sebuah program untuk menghitung luas dan keliling lingkaran! Radius
  lingkaran dimasukkan oleh pengguna.
\end{enumerate}
\end{document}
