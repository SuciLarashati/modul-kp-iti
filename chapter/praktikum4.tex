\documentclass[../main.tex]{subfiles}

\begin{document}
\chapter{Pengulangan Proses - For}
\section{Motivasi}
Pengulangan adalah proses seurutan instruksi yang secara kontinu berulang sampai
sebuah kondisi tercapai. Biasanya, saat sebuah proses berhenti, seperti proses
pengambilan data dan mengubah data, proses tersebut akan diperiksa kondisinya.
Jika kondisinya tidak tercapai, proses urutan instruksi akan diulang kembali
sampai kondisinya tercapai, biasanya dengan masukan yang berbeda. Sebuah
pengulangan adalah konsep penting dalam program.

\section{\eng{For}}
Cara penulisan stuktur \eng{for} adalah :

\begin{minted}{cpp}
for (kondisi awal; kondisi akhir; step){
	pernyataan;
}
\end{minted}

Berikut ini adalah program untuk mencetak bilangan bulat dari 0 hingga 10 dengan
menggunakan struktur \eng{for}:

\cppfile{code/menghitung_10.cpp}

Selain memberikan perintah untuk menampilkan, di dalam struktur \eng{for} kita
juga bisa memberikan perintah lain. Seperti operasi aritmatika, struktur
\eng{if}, dan lain-lain.

Berikut ini adalah program untuk menjumlahkan deret bilangan
\(1 + 2 + 3 + \ldots + n\):

\cppfile{code/jumlah_deret.cpp}

\paragraph{Latihan}
\begin{enumerate}
  \item Buatlah sebuah program untuk menghitung faktorial.
  \item Buatlah sebuah program untuk menghasilkan:

	\begin{minted}[fontsize=\huge]{cpp}
	*
	**
	***
	****
	*****
	\end{minted}
\end{enumerate}

\end{document}
