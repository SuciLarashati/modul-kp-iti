\documentclass[../main.tex]{subfiles}

\begin{document}
\chapter*{Lab Komputer dan Lab Informatika IF/FTI-ITI}
\section*{Tata Tertib Praktikum Komputer}
\begin{enumerate}
\item Penilaian Praktikum ini dilakukan terhadap 3 hal yaitu:
  \begin{enumerate}
  \item Penilaian hasil dari praktikum
  \item Kehadiran
  \item Sikap dan tingkah laku saat praktikum
  \end{enumerate}

\item Dalam praktikum ini peserta harus memperhatikan hal-hal sebagai berikut:
  \begin{enumerate}
  \item Kehadiran peserta praktikum harus 100\%, semua modul praktikum harus
  dilaksanakan. Apabila peserta tidak hadir pada jadwal yang ditentukan, peserta
  harus dilaksanakan pada jam yang lain dengan konsekuensi membayar biaya
  perawatan sebesar Rp@. 5.000,- (lima ribu rupiah) permodul. Peserta yang tidak
  melaksanakan salah satu modul dan tidak mengganti pada jam yang lain, maka
  nilai praktikum modul tersebut adalah 0.
  \item Bagi yang memerlukan praktikum tambahan, dapat menggunakan fasilitas
  laboratorium (bila ada yang kosong) dengan membayar biaya perawatan sebesar
  Rp.\ 1.500,- (seribu lima ratus rupiah) per jam atau bagian dari 1 jam.
  \item Keterlambatan lebih dari 15 menit tidak diperkenankan mengikuti praktikum.
  \item Setiap praktikum, Kartu Praktikum harus dibawa dan diserahakn pada
  asisten yang bertugas.
  \item Duduklah pada tempat yang telah ditentukan sesuai dengan nomor yang
  tertera di Kartu Praktikum.
  \item Tas, buku, \eng{flash disk}, dll harus diletakkan pada tempat yang telah
  disediakan \emph{kecuali Buku Petunjuk Praktikum}.
  \item Peserta harus berpakaian rapih dan tidak diperkenankan memakai sandal.
  \item Ruangan praktikum merupakan ruangan ber-\eng{AC}. Di mana tidak seorang
  pun diperkenankan untuk merokok, membawa makanan, dan minuman.
  \item Peralatan komputer yang ada adalah peralatan yang berharga. Kecerobohan
  peserta yang menyebabkan kerusakan alat harus ditanggung oleh peserta itu sendiri.
  \item Praktikan dilarang mengganti atau mengubah perangkat lunak
  (\eng{Software}) atau kata sandi yang sudah ada.
  \item Selama praktikum berlangsung, jaga kesopanan dan ketenangan supaya tidak
  mengurangi manfaat dari praktikum anda. Sangsi atas pelanggaran ini dapat
  mempengaruhi nilai anda.
  \item Setelah praktikum selesai, harap bersihkan meja praktikum dari
  sampah-sampah dan membuang sampah tersebut pada tempat yang telah disediakan.
  \item Sebelum meninggalkan ruangan, komputer dan monitor yang digunakan harus
  dalam keadaan mati.
  \item Peserta harus meninggalkan ruangan bila ada aba-aba untuk selesainya
  waktu praktikum.
  \item Nilai praktikum merupakan komponen penentu nilai akhir.
  \end{enumerate}
\end{enumerate}

\section*{Tata Tertib Penggunaan Petunjuk Praktikum}
Buku Petunjuk Praktikum adalah \emph{milik Institut}, tidak diberikan tetapi
\emph{dipinjamkan} selama satu semester. Apabila rusak\slash{}hilang maka
peserta dikenakan denda sebesar Rp.\ 10.000,- (sepuluh ribu rupiah) atau
mengganti buku tersebut.
\end{document}
